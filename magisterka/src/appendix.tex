\appendix

\chapter{Zawarto�� za��czonej p�yty}

\chapter{Konfiguracja komputera wykorzystanego do test�w wydajno�ciowych}

  Do test�w u�yto komputera z systemem operacyjnym Ubuntu w wersji 10.10.
  Dok�adn� konfiguracj� komputera, zaprezentowan� poni�ej, uzyskano poleceniem \verb|lshw|.
  Za��czon� jedynie dane istotne z punktu widzenia interpretacji test�w, to znaczy
  informacje o liczbie i wydajno�ci rdzeni procesor�w oraz dane na temat pami�ci RAM.
  
  \begin{verbatim}
   pthink
    description: Notebook
    product: 2089WFY
    vendor: LENOVO
    version: ThinkPad T500
    serial: L3AXL6X
    width: 32 bits
    capabilities: smbios-2.4 dmi-2.4 smp-1.4 smp
  *-core
       description: Motherboard
       product: 2089WFY
       vendor: LENOVO
       physical id: 0
       version: Not Available
       serial: VF28895W17A
     *-firmware
          description: BIOS
          vendor: LENOVO
          physical id: 0
          version: 6FET66WW (2.16 ) (04/22/2009)
          size: 128KiB
          capacity: 8128KiB
          capabilities: pci pcmcia pnp upgrade shadowing escd cdboot bootselect socketedrom edd acpi usb biosbootspecification
    *-cpu:0
          description: CPU
          product: Intel(R) Core(TM)2 Duo CPU     P8600  @ 2.40GHz
          vendor: Intel Corp.
          physical id: 6
          bus info: cpu@0
          version: 6.7.10
          serial: 0001-067A-0000-0000-0000-0000
          slot: None
          size: 2400MHz
          capacity: 2400MHz
          width: 64 bits
          clock: 266MHz
          capabilities: boot fpu fpu_exception wp vme de pse tsc msr pae mce cx8 apic mtrr pge mca cmov pat pse36 clflush dts acpi mmx fxsr sse sse2 ss ht tm pbe nx x86-64 constant_tsc arch_perfmon pebs bts aperfmperf pni dtes64 monitor ds_cpl vmx smx est tm2 ssse3 cx16 xtpr pdcm sse4_1 xsave lahf_lm ida tpr_shadow vnmi flexpriority cpufreq
          configuration: id=0
        *-cache:0
             description: L1 cache
             physical id: a
             slot: Internal L1 Cache
             size: 64KiB
             capacity: 64KiB
             capabilities: synchronous internal write-back instruction
        *-cache:1
             description: L2 cache
             physical id: c
             slot: Internal L2 Cache
             size: 3MiB
             capacity: 3MiB
             capabilities: burst internal write-back unified
        *-logicalcpu:0
             description: Logical CPU
             physical id: 0.1
             width: 64 bits
             capabilities: logical
        *-logicalcpu:1
             description: Logical CPU
             physical id: 0.2
             width: 64 bits
             capabilities: logical
     *-cache
          description: L1 cache
          physical id: b
          slot: Internal L1 Cache
          size: 64KiB
          capacity: 64KiB
          capabilities: synchronous internal write-back data
     *-memory
          description: System Memory
          physical id: 2b
          slot: System board or motherboard
          size: 2GiB
        *-bank:0
             description: SODIMM Synchronous 1066 MHz (0.9 ns)
             product: M471B5673EH1-CF8
             vendor: 80CE
             physical id: 0
             serial: 877BBB48
             slot: DIMM 1
             size: 2GiB
             width: 64 bits
             clock: 1066MHz (0.9ns)
        *-bank:1
             description: SODIMM DDR2 Synchronous 1066 MHz (0.9 ns) [empty]
             physical id: 1
             slot: DIMM 2
             clock: 1066MHz (0.9ns)
     *-cpu:1
          physical id: 1
          bus info: cpu@1
          version: 6.7.10
          serial: 0001-067A-0000-0000-0000-0000
          size: 2401MHz
          capacity: 2401MHz
          capabilities: vmx ht cpufreq
          configuration: id=0
        *-logicalcpu:0
             description: Logical CPU
             physical id: 0.1
             capabilities: logical
        *-logicalcpu:1
             description: Logical CPU
             physical id: 0.2
             capabilities: logical 
  \end{verbatim}
  
  Programy testowe by�y kompilowane przy pomocy kompilatora GCC w wersji 4.4.5.
  Do test�w zosta�y wykorzystane biblioteki z kolekcji Boost w wersji 1.42.0.
