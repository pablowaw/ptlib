
\chapter*{Wprowadzenie}
\addcontentsline{toc}{chapter}{Wprowadzenie}

  Równoległe prowadzenie obliczeń nie jest nowym tematem w informatyce, gdyż liczy sobie już ponad 50 lat\cite{parhist}.
  Trudno wyznaczyć jeden punkt lub jedną pracę, która zapoczątkowała ten kierunek rozwoju informatyki, ale niewątpliwie do jej pionierów należeli tak znani naukowcy jak
  Amdahl (prawo Amdahl-a), Flynn (klasyfikacja Flynna), Dijkstra (problem sekcji krytycznych oraz semafory), Petri (sieci Petriego).
  
  Zrównoleglanie obliczeń jest jednym ze sposobów przyspieszania wydajności systemów komputerowych.
  Szczególnego znaczenia nabrało ostatnimi czasy, ponieważ rozwój technologii mikroprocesorowej dotarł do takiego momentu, że przyspieszanie pojedyńczego układu stało się trudne i dlatego nieopłacalne.
  Stąd obecnie kierunek rozwoju wyznaczany jest przez równoległość, to znaczy umieszczanie w procesorach komputerów wielu układów wykonujących obliczenia równolegle w jednym czasie.
  Takie rozwiązanie teoretycznie pozwala na uzyskanie przyspieszenia wprost proporcjonalnego do liczby układów umieszczonych w procesorze.
  Żeby sie jednak tak stało programy wykonywane na takim procesorze powinny być w stanie wykorzystać te możliwości.
  
  Programy zatem powinny wykorzystywać równoległe prowadzenie obliczeń, jednakże pomimo długo rozwijanej teorii oraz narzędzi wspierających, programowanie równoległe wciaż pozostaje bardzo trudne do zastosowania w praktyce.
  Dzieje się tak, ponieważ wykorzystanie kilku ciągów instrukcji w ramach, których program będzie wykonywany, drastycznie zwiększa złożoność kodu.
  Programista, aby napisać program współbieżny (\cite{barney}) musi w pierwszej kolejności zidentyfikować fragmeny obliczeń, które mogą zostać zrównoleglone.
  Następnie powinien zaprojektować sposób w jaki poszczególne ciągi wykonania się ze sobą komunikują i w jaki sposób są synchronizowane.
  W celu zapewnienia efektywności programu programista powinien uwzględnić w projekcie również balansowanie rozkładu pracy pomiedzy poszczególne wątki wykonania.
  
  Programistę w zmaganiach z pisaniem programów wspierają różnorodne narzędzia, języki programowania specjalnie zaprojektowane do obliczeń równoległych jak Ada lub biblioteki oferujące wykonywanie obliczeń równoległych
  w językach natywnie sekwencyjnych.
  Zazwyczaj te pierwsze wywiązują się ze swojego zadania dobrze, gdyż zostały są dedykowane do programowania równoległego, lecz nie są popularne w zastosowaniach praktycznych.
  Większym problemem jest korzystanie z tych drugich, ponieważ język sekwencyjny zazwyczaj nie pozwala na zaprojektowanie biblioteki mogącej konkurować prostotą i intiucyjnością z językiem do programowani równoległego.
  W praktyce potrzeba skorzystania z bibliotek dla języków sekwencyjnych występuje zdecydowanie częściej, ponieważ są bardzo popularne.
  Niniejsza praca podejmuje próbę stworzenia takiej biblioteki dla języka C++.
  Nadrzędnymi priorytetami podczas projektowania biblioteki Parallel było zapewnienie łatwości pisania programów równoległych (ukrycie niepotrzebnych detali przed programistą) z jednoczesnym pozostawieniem pełnej kontroli 
  nad wykonaniem programu w rękach programisty. Zastosowane rozwiązania powinny zagwarantować wydajność działania biblioteki oraz poprawność.

  Pierwszy rozdział pracy opisuje koncepcję biblioteki, główne zasady nią rządzące oraz porównuje projektowane rozwiązanie do istniejąch rozwiażań.
  Drugi rozdział traktuje o sposobie implementacji biblioteki.
  W trzecim rozdziale zostały przedstawione metody oraz wyniki ewaluacji biblioteki.
  Rozdział ostatni podsumowujący nakreśla możliwości dalszego rozwoju biblioteki.

\section*{Definicje pojęć i skrótów}
\begin{tabular}{ | l | l |}
  \hline
  \textbf{Pojęcie} & \textbf{Definicja} \\ \hline
  Idiom C++& Konstrukcja języka C++, która często pojawia się w kodzie lub projektach doświadczonych programistów C++. Stosowanie jej uważane jest za dobrą praktykę.\\ \hline
  
  \hline
\end{tabular} 