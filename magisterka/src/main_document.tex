%
% Niniejszy plik stanowi przykład formatowania pracy magisterskiej na
% Wydziale MIM UW.  Szkielet użytych poleceñ można wykorzystywać do
% woli, np. formatujac wlasna prace.
%
% Zawartosc merytoryczna stanowi oryginalnosiagniecie
% naukowosciowe Marcina Wolinskiego.  Wszelkie prawa zastrzeżone.
%
% Copyright (c) 2001 by Marcin Woliñski <M.Wolinski@gust.org.pl>
% Poprawki spowodowane zmianami przepisów - Marcin Szczuka, 1.10.2004
% Poprawki spowodowane zmianami przepisow i ujednolicenie 
% - Seweryn Karłowicz, 05.05.2006
% dodaj opcję [licencjacka] dla pracy licencjackiej
\documentclass{pracamgr}

\usepackage{polski}

%Jesli uzywasz kodowania polskich znakow ISO-8859-2 nastepna linia powinna byc 
%odkomentowana
\usepackage[utf8]{inputenc}
%Jesli uzywasz kodowania polskich znakow CP-1250 to ta linia powinna byc 
%odkomentowana
%\usepackage[cp1250]{inputenc}


\author{Paweł Tryfon}

\nralbumu{248444}

\title{Biblioteka do równoległego obliczania wyrażeń w języku C++}

\tytulang{A library for parallel expression evaluation in C++}

%kierunek: Matematyka, Informatyka, ...
\kierunek{Informatyka}

% Praca wykonana pod kierunkiem:
% (podać tytuł/stopieñ imię i nazwisko opiekuna
% Instytut
% ew. Wydział ew. Uczelnia (jeżeli nie MIM UW))
\opiekun{dra Marcina Benke\\
  Zakład Logiki Stosowanej
  }

% miesiąc i~rok:
\date{Maj 2011}

%Podać dziedzinę wg klasyfikacji Socrates-Erasmus:
\dziedzina{ 
%11.0 Matematyka, Informatyka:\\ 
%11.1 Matematyka\\ 
%11.2 Statystyka\\ 
11.3 Informatyka\\ 
%11.4 Sztuczna inteligencja\\ 
%11.5 Nauki aktuarialne\\
%11.9 Inne nauki matematyczne i informatyczne
}

%Klasyfikacja tematyczna wedlug AMS (matematyka) lub ACM (informatyka)
\klasyfikacja{D. Software\\
  D.3. Programming languages\\
  D.3.3. Language Constructs and Features\\
  Subject: Concurrent Programming Constructs}

% Słowa kluczowe:
\keywords{obliczenia równoległe, C++, wielowątkowość, leniwe wyliczanie, programowanie generyczne}

% Tu jest dobre miejsce na Twoje własne makra i~środowiska:
%\newtheorem{defi}{Definicja}[section]

% koniec definicji

\begin{document}
\maketitle

%tu idzie streszczenie na strone poczatkowa
\begin{abstract}
  Obecne architektury pozwalają na przyspieszanie działania programów dzięki ich wykonywaniu jednocześnie na kilku procesorach.
  Jednakże skorzystanie z tej możliwości przedstawia istotną trudność dla programistów, gdyż programy wielowątkowe w swojej strukturze bardzo różnią się od programów sekwencyjnych.
  Projektowanie i implementacja programów wykorzystujących współbieżność jest znacznie bardziej czasochłonna oraz wymaga wyższych kwalifikacji.
  Niniejsza praca podejmuje próbę stworzenia biblioteki, która ułatwiłaby zadanie programowania programów wielowątkowych.  
  Głównym priorytetem byłu umożliwienie programiście zrównoleglania obliczeñ w zwięzły, zrozumiały i prosty sposób.    
  Obecnie nie istnieje w języku C++ żadna biblioteka oferująca taką funcjonalność.
  Praca przedstawia model prowadzenia obliczeñ równoległych w języku C++ oraz prezentuje proponowaną implementację.
  W pracy zostały szczegółowo opisane problemy, które zostały rozwiązane podczas projektowania biblioteki, 
  takie jak: mechanizm przekazywania wyrażeñ do wyliczenia równoległego, sposób prowadzenia równoległych obliczeñ, 
  sposób zwracania wyników obliczeñ oraz metody zapobiegania problemom związanym z prowadzeniem równoległych obliczeñ.
\end{abstract}

\tableofcontents
%\listoffigures
%\listoftables


\chapter*{Wprowadzenie}
\addcontentsline{toc}{chapter}{Wprowadzenie}

  Równoległe prowadzenie obliczeń nie jest nowym tematem w informatyce, gdyż liczy sobie już ponad 50 lat\cite{parhist}.
  Trudno wyznaczyć jeden punkt lub jedną pracę, która zapoczątkowała ten kierunek rozwoju informatyki, ale niewątpliwie do jej pionierów należeli tak znani naukowcy jak
  Amdahl (prawo Amdahl-a), Flynn (klasyfikacja Flynna), Dijkstra (problem sekcji krytycznych oraz semafory), Petri (sieci Petriego).
  
  Zrównoleglanie obliczeń jest jednym ze sposobów przyspieszania wydajności systemów komputerowych.
  Szczególnego znaczenia nabrało ostatnimi czasy, ponieważ rozwój technologii mikroprocesorowej dotarł do takiego momentu, że przyspieszanie pojedyńczego układu stało się trudne i dlatego nieopłacalne.
  Stąd obecnie kierunek rozwoju wyznaczany jest przez równoległość, to znaczy umieszczanie w procesorach komputerów wielu układów wykonujących obliczenia równolegle w jednym czasie.
  Takie rozwiązanie teoretycznie pozwala na uzyskanie przyspieszenia wprost proporcjonalnego do liczby układów umieszczonych w procesorze.
  Żeby sie jednak tak stało programy wykonywane na takim procesorze powinny być w stanie wykorzystać te możliwości.
  
  Programy zatem powinny wykorzystywać równoległe prowadzenie obliczeń, jednakże pomimo długo rozwijanej teorii oraz narzędzi wspierających, programowanie równoległe wciaż pozostaje bardzo trudne do zastosowania w praktyce.
  Dzieje się tak, ponieważ wykorzystanie kilku ciągów instrukcji w ramach, których program będzie wykonywany, drastycznie zwiększa złożoność kodu.
  Programista, aby napisać program współbieżny (\cite{barney}) musi w pierwszej kolejności zidentyfikować fragmeny obliczeń, które mogą zostać zrównoleglone.
  Następnie powinien zaprojektować sposób w jaki poszczególne ciągi wykonania się ze sobą komunikują i w jaki sposób są synchronizowane.
  W celu zapewnienia efektywności programu programista powinien uwzględnić w projekcie również balansowanie rozkładu pracy pomiedzy poszczególne wątki wykonania.
  
  Programistę w zmaganiach z pisaniem programów wspierają różnorodne narzędzia, języki programowania specjalnie zaprojektowane do obliczeń równoległych jak Ada lub biblioteki oferujące wykonywanie obliczeń równoległych
  w językach natywnie sekwencyjnych.
  Zazwyczaj te pierwsze wywiązują się ze swojego zadania dobrze, gdyż zostały są dedykowane do programowania równoległego, lecz nie są popularne w zastosowaniach praktycznych.
  Większym problemem jest korzystanie z tych drugich, ponieważ język sekwencyjny zazwyczaj nie pozwala na zaprojektowanie biblioteki mogącej konkurować prostotą i intiucyjnością z językiem do programowani równoległego.
  W praktyce potrzeba skorzystania z bibliotek dla języków sekwencyjnych występuje zdecydowanie częściej, ponieważ są bardzo popularne.
  Niniejsza praca podejmuje próbę stworzenia takiej biblioteki dla języka C++.
  Nadrzędnymi priorytetami podczas projektowania biblioteki Parallel było zapewnienie łatwości pisania programów równoległych (ukrycie niepotrzebnych detali przed programistą) z jednoczesnym pozostawieniem pełnej kontroli 
  nad wykonaniem programu w rękach programisty. Zastosowane rozwiązania powinny zagwarantować wydajność działania biblioteki oraz poprawność.

  Pierwszy rozdział pracy opisuje koncepcję biblioteki, główne zasady nią rządzące oraz porównuje projektowane rozwiązanie do istniejąch rozwiażań.
  Drugi rozdział traktuje o sposobie implementacji biblioteki.
  W trzecim rozdziale zostały przedstawione metody oraz wyniki ewaluacji biblioteki.
  Rozdział ostatni podsumowujący nakreśla możliwości dalszego rozwoju biblioteki.

\section*{Definicje pojęć i skrótów}
\begin{tabular}{ | l | l |}
  \hline
  \textbf{Pojęcie} & \textbf{Definicja} \\ \hline
  Idiom C++& Konstrukcja języka C++, która często pojawia się w kodzie lub projektach doświadczonych programistów C++. Stosowanie jej uważane jest za dobrą praktykę.\\ \hline
  
  \hline
\end{tabular} 


\chapter{Koncepcja biblioteki}\label{r:koncepcja}

  W tym rozdziale zostaną przedstawione główne założenia biblioteki Parallel na tle istnijących już modeli programowania równoległego, dostępnych w języku C++.

\section{Cele biblioteki}

  Tworzeniu biblioteki Parallel przyświecały bardzo konkretne cele, których ideą przewodnią było ułatwienie wykorzystywania obliczeń równoległych w programach.
  Wymienionym poniżej celom było podporządkowane projektowanie API i implementacja biblioteki.

\subsection{Wysoka efektywność}

  Jednym z głównych powodów stosowania zrównoleglania obliczeń jest przyspieszanie ich wykonania. Dlatego sama biblioteka do zrównoleglania powinna działać szybko.
  Niedopuszczalną byłaby sytuacja, gdyby program współbieżny wykonywał się wolniej niż jego sekwencyjny odpowiednik.
  Biblioteka Parallel będzie biblioteką ogólnego zastosowania, przy pomocy, które będzie możliwe prowadzenie dowolnych obliczeń.
  Niemożliwe jest takie napisanie biblioteki ogólnej, żeby w każdej sytuacji działała bardzo wydajnie
  Dlatego, oprócz szybkiego działania mechanizmów wbudowanych w bibliotekę, niezbędne jest pozwolenie programiście na podejmowanie decyzji o takim prowadzeniu obliczeń, że ich wykonanie przy użyciu biblioteki Parallel będzie efektywne..
  
\subsection{Zwiększenie produktywności programisty}
  Problem z efektywnością programisty w przypadku pisania programów równoległych polega na tym, że takie programy są trude do pisania, stąd wymagają znacznych nakładów czasowych.
  Zrównoleglenie choćby niewielkiego fragmentu programu wymaga często znacznie więcej czasu niż napisanie jego sekwencyjnego odpowiednika.
  Być może dlatego obliczenia równoległe wykorzystywane są wyłącznie wtedy, gdy już nie ma innego sposobu osiągnięcia niezbędnego minimum wydajności programu.
  Biblioteka Parallel celuje w zmianę tego stanu rzeczy, dzięki wprowadzeniu modelu programowania równoległego, który będzie tak samo intuicyjny jak programowanie sekwencyjne.
  Dzięki czemu napisanie kodu, który działa współbieżnie, będzie prawie tak samo szybkie jak kodu sekwencyjnego, co pozwoliłoby uzyskać programistom szybsze programy przy tej samej produktywności.

\subsection{Czytelność kodu}

  Tym, co najbardziej utrudnia zrozumienie programów współbieżnych jest konieczność zrozumienia zależności pomiędzy odrębnymi równolegle działającymi częściami programu.
  Zazwyczaj te zależnośći dotyczą miejsc w kodzie, które są od siebie stosunkowo odległe.
  Mnogość niejawnych zależności i przeplotów wykonań programu sprawiają, że nawet pozornie proste operacje są trudne do poprawnego zaprogramowania.
  Jednym z bardziej wymownych przykładów popierających to stwierdzenie jest problem implementacji semafora uogólnionego przy pomocy semaforów binarnych \cite{gensem}.
  Stąd celem, który został postawiony przed biblioteką Parallel było ukrycie do takiego stopnia, do jakiego to możliwe, obecności równoległości w kodzie.
  Najważniejsze jest to, że struktura programu napisanego przy pomocy biblioteki Parallel nie powinna istotnie różnić się od struktury programu sekwencyjnego.
  Pozwoli to na uzyskanie kodu, który będzie znacznie łatwiej zrozumieć.

\subsection{Transparencja}

  Biblioteka Parallel powinna udostępniać programiście wgląd w to, w jaki sposób oblczenia równoległe będą prowadzone.
  Dzięki temu programista będzie mógł uwzględnić podczas programowania ograniczenia, które wynikają z konstrukcji biblioteki.
  Między innymi będzie mógł dostosować wielkość zlecanych fragmentów obliczeń (ziarnistość obliczeń), tak aby zmaksymalizować wydajność programu.
  
\subsection{Abstrakcja}

  Abstrakcja ukrywa niepotrzebne szczegóły implementacji przed programistą, co pozwala na zwiększenie jego produktywności.
  Używając biblioteki Parallel programista nie będzie musiał uczyć się skomplikowanego API, ponieważ uogólnione API będzie proste.

\subsection{Ograniczenie konieczności korzystania z mechanizmów komunikacji i synchronizacji procesów równoległych}

  Projektowawnie komunikacji i synchronizacji w programach współbieżnych jest czymś, co decyduje o fakcie, że programowanie równoległe jest tak trudnym zadaniem.
  Celem biblioteki Parallel jest zdjęcie w znacznym stopniu tego obciążenia z programisty.
  Komunikacja pomiędzy różnymi wątkami wykonania będzie koordynowana przez bibliotekę.
  Biblioteka nie może wyręczyć jednak programisty we wszystkim, ochrona spójności struktur danych pozostanie w rękach programisty.
  
\subsection{Przenaszalność}
  
  Jest to bardzo istotna cecha biblioteki, dzięki której kod pisany przy użyciu biblioteki Parallel będzie mógł być kompilowany i wykonywany na dowolnych platformach.
  Zostanie to osiągnięte dzięki zaprogramowaniu biblioteki w pełnej zgodności z nowym standardem języka C++ (standard C++0x).
  Użycie nowego standardu jest niezbędne ze względu na zaawansowane konstrukcje językowe potrzebne do zaprogramowania biblioteki Parallel.
  Konsekwencją tego będzie niezgodność biblioteki z wcześniejszymi standardami języka C++, ale umożliwi stworzenie lepszego, bardziej czytelnego kodu biblioteki przy użyciu nowoczesnych technik programowania w C++.

\section{Model programowania w bibliotekce Parallel}

  Model programowania przy użyciu biblioteki był tak projekowany, aby spełnić cele wyszczególnione w poprzedniej sekcji.
  Przedstawię poniżej inspirację oraz wynik końcowy pracy koncepcyjnej nad biblioteką, jak i uzasadnienie podjętych decyzji.

\subsection{Inspiracja}

  Powstanie biblioteki zostało zainspirowane biblioteką do prowadzenia obliczeń równoległych w języku Haskell, Parallel Haskell\cite{parhas}.
  Ta biblioteka pozwala w sposób bardzo intuicyjny obliczać dwa wyrażenia równolegle.
  Oto przykład funkcji obliczającej w sposób równoległy n-tą liczbę Fibonacciego:
  \begin{verbatim}
    import Control.Parallel

    nfib :: Int -> Int
    nfib n | n <= 1 = 1
       | otherwise = par n1 (seq n2 (n1 + n2))
                     where n1 = nfib (n-1)
                           n2 = nfib (n-2)
  \end{verbatim}
  
  Analogiczny program sekwencyjny wyglądałby następująco:
  \begin{verbatim}
    import Control.Parallel

    nfib :: Int -> Int
    nfib n | n <= 1 = 1
       | otherwise = (n1 + n2)
                     where n1 = nfib (n-1)
                           n2 = nfib (n-2)
  \end{verbatim}
  
  W Haskellu funkcja \verb|par| wskazuje, że wyliczenie równoległe może być korzystne i w czasie wykonania podejmowana jest decyzja o sposobie wyliczenia.
  Ta konstrukcja pokazuje z jak zadziwiającą prostotą można pisać programy równoległe.
  
\subsection{Zlecanie obliczeń}

  UWAGA!!! We wszystkich przedstawionych poniżej fragmentach kodu przyjęto, że zostały napisane z deklaracją \verb|using parallel;|.\\

  Zlecanie równoległego wykonania obliczeń powinno jak najmniej ingerować w sekwencyjny kod programu dla zachowania jego intuicyjności.
  Do oznaczenia wyrażenia, które ma zostać wykonane równolegle wykorzystywana jest funkcja \verb|evaluate|, przyjmująca jako argument wyrażenie do wykonania.
  Obliczenie przekazanego wyrażenia odbędzie się równolegle.
  W tej chwili czytelnik dobrze zaznajomiony z semantyką języka C++ zapewne zauważył problem związany z przekazaniem wyrażenia do wykonania.
  Jeśli byłoby ono przekazane w standardowej formie w języku C++, to zostałoby wyliczone i dopiero wtedy przekazane do funkcji \verb|evaluate|, ponieważ w języku C++ wszystkie wyrażenia wyliczane są gorliwie.
  Dlatego niezbędne było zaprojektowanie uleniwionego sposobu przekazywania wyrażenia do funkcji \verb|evaluate|.
  
\subsubsection{Leniwe wyrażenia w języku C++}

  W pierwszej chwili stworzenie leniwego wyrażenia w języku C++ może wydawać się niemożliwe lub bardzo trudne. 
  Domyśla semantyka jest gorliwe, nie ma żadnych słów kluczowych pozwalających na dodanie leniwości, C++ nie pozwala również na rozszerzenie składni języka.
  Wskazówkę do rozwiazanie problemu leniwych wyrażeń w języku C++ znalazłem w książce \textit{More C++ Idioms}\cite{idioms}, która przedstawia idiom C++ szablonu wyrażenia.
  
\paragraph{Idiom C++ szablonu wyrażenia}
  
  Idiom szablonu wyrażenia 

\subsection{Wykonanie zadań}

\subsection{Sposób przekazywania wyniku}

\section{Inne modele programowania równoległego w języku C++}

\section{Analiza cech biblioteki}

%   Programista w celu wykorzystania możliwości równoległego musi dokonać wyboru modelu programowania równoległego, narzędzi, które ten model wspierają, 
%   a następnie do dokonanych decyzji zaprojektować program.
%   Proces projektowania programu równoległego jest znacznie bardziej skomplikowany niż programu sekwencyjnego, ze względu na to, 
%   że zazwyczaj narzędzia do programowania równoległego narzucają strukturę programu, która w znacznym stopniu różni się od programu sekwencyjengo, 
%   nie jest tak naturalna, a przez to trudniejsza do wykonania i zrozumienia.
%   Dwa podstawowe problemy, które towarzyszą wyborowi technologii do programowania równoległego, to poziom abstrakcji oraz wydajność.
%   Poziom abstrakcji



\chapter{Opis implementacji}\label{r:implementacja}

\chapter{Ewaluacja biblioteki}\label{r:ocena}

\section{Teretyczne podstawy}
Prawo Amdahla - o ile można przyspieszyć program.
\chapter{Podsumowanie}\label{r:podsumowanie}

\appendix


\begin{thebibliography}{99}
\addcontentsline{toc}{chapter}{Bibliografia}

\bibitem[Ben-Ari06]{benari} Mordechai Ben-Ari, \textit{Principles of Concurrent and Distributed Programming,
  Second Edition}, Addison-Wesley, 2006.

\bibitem[ParC++]{parc++} Cameron Hughes, Tracey Hughes, \textit{Parallel \& Distributed Programming using C++},
  Adison-Wesley, 2003.

\bibitem[TemG]{temguide} David Vandevoorde, Nicolai M. Josuttis, \textit{C++ Templates: The Complete Guide},
  Addison Wesley, 2002.

\bibitem[BerLand]{berland} Krste Asanovíc, Rastislav Bodik, Bryan Catanzaro, Joseph Gebis,
  Parry Husbands, Kurt Keutzer, David Patterson,
  William Plishker, John Shalf, Samuel Williams, and Katherine Yelick,
  \textit{The Landscape of Parallel Computing Resarch: A view from Berkley},
  Electrical Engineering and Computer Science, University of California at Berkeley, 2006,
  Technical Report No. UCB/EECS-2006-183,
  \texttt{http://www.eecs.berkeley.edu/Pubs/TechRpts/2006/EECS-2006-183.html}.

\bibitem[ParHist]{parhist} Wilson, Gregory V Virginia, \textit{The History of the Development of Parallel Computing}, Tech/Norfolk State University, Interactive Learning with a Digital Library in Computer Science, 1994.

\bibitem[Barney]{barney} Blaise Barney, Lawrence Livermore National Laboratory, \textit{Introduction to Parallel Computing}, Livermore Computing, \texttt{https://computing.llnl.gov/tutorials/tutorials/parallel\_comp/}.

\bibitem[Foster]{foster} Ian Foster, \textit{Designing and Building Parallel Programs}, Addison-Wesley, 1995.

\bibitem[GenSem]{gensem} John A. Trono,	William E. Taylor, \textit{Further comments on "A Correct and Unrestrictive Implementation of General Semaphores"}, ACM SIGOPS Operating Systems Review, Volume 34 Issue 3, July 2000.

\bibitem[HasRef]{parhas} \textit{The Glorious Glasgow Haskell Compilation System User's Guide, Version 6.6}, \texttt{http://www.haskell.org/ghc/docs/6.6/html/users\_guide/index.html}.

\bibitem[Proto]{proto} Eric Niebler, Boost.Proto Library \texttt{http://www.boost.org/doc/libs/release/libs/proto/index.html}.

\bibitem[Idioms]{idioms} Sumant Tambe (as the initiator and the lead contributor) and many other authors, \textit{More C++ Idioms}, Wikibooks, \texttt{http://en.wikibooks.org/wiki/More\_C\%2B\%2B\_Idioms}.

\bibitem[ExpTem]{exptem} Klaus Kreft, Angelika Langer, \textit{An Introduction to the Principles of Expression Templates}, C/C++ Users Journal, March 2003, \texttt{http://www.angelikalanger.com/Articles/Cuj/ExpressionTemplates/ExpressionTemplates.htm}.

\bibitem[DisSys]{dissys} Andrew Tanenbaum, Marteen van Steen, \textit{Distributed Systems}, Prentice Hall, 2002.

\bibitem[SmtPool]{smartpool} Ami Bar, \textit{Smart Thread Pool}, \texttt{http://www.codeproject.com/KB/threads/smartthreadpool.aspx}.

\bibitem[ThdPool]{threadpool} Brian Goetz, \textit{Thread pools and work queues}, \texttt{http://www.ibm.com/developerworks/java/library/j-jtp0730/index.html}.
\end{thebibliography}

\end{document}


%%% Local Variables:
%%% mode: latex
%%% TeX-master: t
%%% coding: latin-2
%%% End:
